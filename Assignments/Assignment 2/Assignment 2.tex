\documentclass[fleqn, 12pt]{article}

\usepackage[margin=1in]{geometry}
\usepackage{mathtools, amssymb} % math %
\usepackage{tabularx, multirow} % tables %
\usepackage{listings} % code %
\usepackage{graphicx}
\usepackage{enumerate}
\usepackage{adjustbox}

%Center image and diagrams
\adjustboxset*{center}

\usepackage[T1]{fontenc}
\usepackage[protrusion=true,expansion=true]{microtype}

% fix padding on code indents
\lstset{
    xleftmargin=-25pt,
    frame=single,
    framexleftmargin=-25pt
}

% Define ceiling and floor functions
\DeclarePairedDelimiter\ceil{\lceil}{\rceil}
\DeclarePairedDelimiter\floor{\lfloor}{\rfloor}

% Fix indents
\newlength\tindent
\setlength{\tindent}{\parindent}
\setlength{\parindent}{0pt}

\begin{document}

\noindent
CSI 2110 \hfill Fall 2015\\
Computer Science \hfill University of Ottawa
\begin{center}
{\bf Assignment \#2}\\
Matt Langlois - 7731813\\
October 25\\ \vspace{24pt}
\end{center}

\hrule

\vspace{12pt}

\section*{Question 1}

\begin{enumerate}[a)]
\item 
    20, 15, 10, 5, 3, 7, 13, 12, 19, 16, 35, 25, 40, 38
\item 
    3, 7, 5, 12, 13, 10, 16, 19, 15, 25, 38, 40, 35, 20
\item 
    Searches for the largest value in a binary tree.
    \begin{lstlisting}[language=java, linewidth=4in]
    findMax(Node n) {
        if (n.hasRightChild()) {
            return findMax(n.rightChild());
        }
        return n;
    }
    \end{lstlisting}
\item \emph{Insert element:} 21 \\
    \adjincludegraphics[width=0.6\textwidth]{1d.pdf}
\item
    \begin{enumerate}[\text{Case} 1:]
    \item Replace 15 with the left most node of the right subtree. Then remove the leftmost node of the right subtree.\\
        \adjincludegraphics[width=0.6\textwidth]{1e1.pdf}
    \item Replace 15 with the right most node of the left subtree. Any children of the right most node become children of the right most node's parent.\\
        \adjincludegraphics[width=0.6\textwidth]{1e2.pdf}
    \end{enumerate}
\end{enumerate}

\section*{Question 2}

\begin{enumerate}[a)]
\item
    \begin{enumerate}[Step 1:]
    \item Replace node with the last node in the heap
        \begin{center}
            \begin{tabular}{|*{10}{c|}}
                \hline
                \underline{1} & 14 & 10 & 8 & 7 & 9 & 3 & 2 & 4\\
                \hline
            \end{tabular}
        \end{center}
        \adjincludegraphics[width=0.6\textwidth]{2a1.pdf}
        

    \item Downheap while the children are larger
        \begin{center}
            \begin{tabular}{|*{10}{c|}}
                \hline
                \underline{14} & \underline{1} & 10 & 8 & 7 & 9 & 3 & 2 & 4\\
                \hline
            \end{tabular}
        \end{center}
        \adjincludegraphics[width=0.6\textwidth]{2a2.pdf}
        
    \item Downheap while the children are larger
        \begin{center}
            \begin{tabular}{|*{10}{c|}}
                \hline
                14 & \underline{8} & 10 & \underline{1} & 7 & 9 & 3 & 2 & 4\\
                \hline
            \end{tabular}
        \end{center}
        \adjincludegraphics[width=0.6\textwidth]{2a3.pdf}
    
    \item Downheap to become a leaf node
        \begin{center}
            \begin{tabular}{|*{10}{c|}}
                \hline
                14 & 8 & 10 & \underline{4} & 7 & 9 & 3 & 2 & \underline{1}\\
                \hline
            \end{tabular}
        \end{center}
        \adjincludegraphics[width=0.6\textwidth]{2a4.pdf}
    \end{enumerate}

\item Insert element 3 at the left most position on the empty row. No further changes are required as the $7\ge3$ property of the max heap is satisfied
        \begin{center}
            \begin{tabular}{|*{11}{c|}}
                \hline
                16 & 14 & 10 & 8 & 7 & 9 & 3 & 2 & 4 & 1 & \underline{3}\\
                \hline
            \end{tabular}
        \end{center}
        \adjincludegraphics[width=0.6\textwidth]{2b.pdf}
        
\item
    \begin{enumerate}[Step 1:]
    \item Insert at left most node in the empty row
        \begin{center}
            \begin{tabular}{|*{11}{c|}}
                \hline
                16 & 14 & 10 & 8 & 7 & 9 & 3 & 2 & 4 & 1 & \underline{15}\\
                \hline
            \end{tabular}
        \end{center}
        \adjincludegraphics[width=0.6\textwidth]{2c1.pdf}
        

    \item Upheap 15 until max heap property $parent \ge child$ is met
        \begin{center}
            \begin{tabular}{|*{11}{c|}}
                \hline
                16 & 14 & 10 & 8 & \underline{15} & 9 & 3 & 2 & 4 & 1 & \underline{7}\\
                \hline
            \end{tabular}
        \end{center}
        \adjincludegraphics[width=0.6\textwidth]{2c2.pdf}
        
    \item Upheap once more to satisfy the max-heap property.
        \begin{center}
            \begin{tabular}{|*{11}{c|}}
                \hline
                16 & \underline{15} & 10 & 8 & \underline{14} & 9 & 3 & 2 & 4 & 1 & 7\\
                \hline
            \end{tabular}
        \end{center}
        \adjincludegraphics[width=0.6\textwidth]{2c3.pdf}
    \end{enumerate}

\item Calculate the height of a 2000 element heap:
    \begin{align*}
      h(n) &= \floor{log_2 (n)} \\
      h(2000) &= \floor{log_2 (2000)}\\
      h(2000) &= 10
    \end{align*}
    $\therefore$ the height of a 2000 node heap is 10.

\end{enumerate}

\end{document}