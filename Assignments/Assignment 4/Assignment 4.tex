\documentclass[fleqn, 12pt]{article}

\usepackage[margin=1in]{geometry}
\usepackage{mathtools, amssymb} % math %
\usepackage{tabularx, multirow} % tables %
\usepackage{listings} % code %
\usepackage{graphicx}
\usepackage{enumerate}
\usepackage{adjustbox}

%Center image and diagrams
\adjustboxset*{center}

\usepackage[T1]{fontenc}
\usepackage[protrusion=true,expansion=true]{microtype}

% fix padding on code indents
\lstset{
    xleftmargin=-25pt,
    frame=single,
    framexleftmargin=-25pt
}

% Define ceiling and floor functions
\DeclarePairedDelimiter\ceil{\lceil}{\rceil}
\DeclarePairedDelimiter\floor{\lfloor}{\rfloor}

% Fix indents
\newlength\tindent
\setlength{\tindent}{\parindent}
\setlength{\parindent}{0pt}

\begin{document}

\noindent
CSI 2110 \hfill Fall 2015\\
Computer Science \hfill University of Ottawa
\begin{center}
{\bf Assignment \#4}\\
Matt Langlois - 7731813\\
December 7\\ \vspace{24pt}
\end{center}

\hrule

\vspace{12pt}

\section*{Question 1}

\begin{enumerate}[a)]
\item 2-4 Tree with 13 3-nodes\\
    \adjincludegraphics[width=0.6\textwidth]{1a1.pdf}
\item 
    After 1 minimum insertion: \\
    \adjincludegraphics[width=0.6\textwidth]{1b1.pdf}
    After 2 minimum insertions: \\
    \adjincludegraphics[width=0.6\textwidth]{1b2.pdf}
    After 3 minimum insertions: \\
    \adjincludegraphics[width=0.6\textwidth]{1b3.pdf}
    \newpage
    After 4 minimum insertions: \\ 
    \adjincludegraphics[width=0.6\textwidth]{1b4.pdf}
    After 5 minimum insertions: \\
    \adjincludegraphics[width=0.6\textwidth]{1b5.pdf}
\item 
    After 1 max deletion: \\
    \adjincludegraphics[width=0.6\textwidth]{1c1.pdf}
    After 2 max deletions: \\
    \adjincludegraphics[width=0.6\textwidth]{1c2.pdf}
    After 3 max deletions: \\
    \adjincludegraphics[width=0.6\textwidth]{1c3.pdf}
    After 4 max deletions: \\
    \adjincludegraphics[width=0.6\textwidth]{1c4.pdf}
    After 5 max deletions: \\
    \adjincludegraphics[width=0.6\textwidth]{1c5.pdf}
\end{enumerate}


\section*{Question 2}

\begin{enumerate}[a)]
\item
    DFS Traversal order: $v5 \rightarrow v4 \rightarrow v2 \rightarrow v4 \rightarrow v3 \rightarrow v1 \rightarrow v10 \rightarrow v9 \rightarrow v8 \rightarrow v6 \rightarrow v7$
    
    \begin{center}
        \begin{tabular}{ | l | l |}
            \hline
            Discovery Edges & Back Edges \\ \hline
            $(v5,v4)$ & $(v10,v3)$ \\ \hline
            $(v4,v2)$ & $(v6,v3)$ \\ \hline
            $(v4,v3)$ & $(v6,v5)$ \\ \hline
            $(v3,v1)$ & $(v7,v8)$ \\ \hline
            $(v1,v10)$ & \\ \hline
            $(v10,v9)$ & \\ \hline
            $(v9,v8)$ & \\ \hline
            $(v8,v6)$ & \\ \hline
            $(v6,v7)$ & \\
            \hline
        \end{tabular}
    \end{center}
    
\item
    BFS Traversal Levels:

    \begin{center}
        \begin{tabular}{ | l | l | l | l | }
            \hline
            Level & Nodes & Discovery Edges & Cross Edges \\ \hline
            L1 & v4 & $(v5,v4)$ & \\ \cline{2-4}
            & v6 & $(v5,v6)$ & \\ \hline
            L2 & v2 & $(v4,v2)$ & $(v6, v3)$ \\ \cline{2-4}
            & v3 & $(v4,v3)$ & \\ \cline{2-4}
            & v7 & $(v6,v7)$ & \\ \cline{2-4}
            & v8 & $(v6,v8)$ & \\ \hline
            L3 & v1 & $(v3,v1)$ & $(v7, v8)$ \\ \cline{2-4}
            & v10 & $(v3,v10)$ & \\ \cline{2-4}
            & v9 & $(v8,v9)$ & \\ \hline
            L4 & & & $(v1,v10)$ \\ \cline{2-4}
            & & & $(v10,v9)$ \\
            \hline
        \end{tabular}
    \end{center}

\end{enumerate}

\newpage

\section*{Question 3}

\begin{enumerate}[a)]
\item Dijkstra's algorithm trace:
    \begin{center}
        \begin{tabular}{ |*{12}{l|}}
            \hline
            New Vertex & S & P & U & X & Q & Y & V & R & W & T & New Edge \\ \hline
            S & 0\checkmark & 6 & 3* & $\infty$ & $\infty$ & $\infty$ & $\infty$ & $\infty$ & $\infty$ & $\infty$ & \\ \hline
            U & 0\checkmark & 6 & 3\checkmark & 4* & $\infty$ & $\infty$ & 6 & $\infty$ & $\infty$ & $\infty$ & $(S,U)$ \\ \hline
            X & 0\checkmark & 6* & 3\checkmark & 4\checkmark & 12 & 10 & 6 & $\infty$ & $\infty$ & $\infty$ & $(U,X)$ \\ \hline
            P & 0\checkmark & 6\checkmark & 3\checkmark & 4\checkmark & 11 & 10 & 6* & $\infty$ & $\infty$ & $\infty$ & $(S,P)$ \\ \hline
            V & 0\checkmark & 6\checkmark & 3\checkmark & 4\checkmark & 11 & 10* & 6\checkmark & $\infty$ & 10 & $\infty$ & $(U,V)$ \\ \hline
            Y & 0\checkmark & 6\checkmark & 3\checkmark & 4\checkmark & 11 & 10\checkmark & 6\checkmark & 11 & 10* & $\infty$ & $(X,Y)$ \\ \hline
            W & 0\checkmark & 6\checkmark & 3\checkmark & 4\checkmark & 11* & 10\checkmark & 6\checkmark & 11 & 10\checkmark & 15 & $(V,W)$ \\ \hline
            Q & 0\checkmark & 6\checkmark & 3\checkmark & 4\checkmark & 11\checkmark & 10\checkmark & 6\checkmark & 11* & 10\checkmark & 15 & $(P,Q)$ \\ \hline
            R & 0\checkmark & 6\checkmark & 3\checkmark & 4\checkmark & 11\checkmark & 10\checkmark & 6\checkmark & 11\checkmark & 10\checkmark & 13* & $(Y,R)$ \\ \hline
            T & 0\checkmark & 6\checkmark & 3\checkmark & 4\checkmark & 11\checkmark & 10\checkmark & 6\checkmark & 11\checkmark & 10\checkmark & 13\checkmark & $(R,T)$ \\
            \hline
        \end{tabular}
    \end{center}
    Minimum Spanning Tree:\\
    \adjincludegraphics[width=0.35\textwidth]{3a1.pdf}
\item SUXYRT is the shortest path from S to T.
\end{enumerate}

\section*{Question 4}

\begin{enumerate}[a)]
\item Insertion into a hash table using linear probing with the function $h(X)=X\mod11$ on the following elements: 3, 32, 46, 58, 57, 26, 17, 74.
    \begin{center}
        \begin{tabular}{|*{12}{c|}}
            \hline
            $h(X)$ & 0 & 1 & 2 & 3 & 4 & 5 & 6 & 7 & 8 & 9 & 10\\ \hline
            x & & & 46 & 3 & 58 & 57 & 26 & 17 & 74 &  & 32\\
            \hline
        \end{tabular}
    \end{center}

\item Four elements would have to be probed before we find 57. This is because $57\mod11=2$ but 57 is found at the 5th position.

\item Eight elements would have to be probed before we realize 68 is not in the table. This is because $68\mod11=2$ but the first empty element is found at the 9th position.

\item
    \begin{enumerate}[i)]
    \item Insertion into a hash table using linear probing with the double hash function $h_2(X)=7-(X\mod7)$ on the following elements: 3, 32, 46, 58, 57, 26, 17, 74.
    \begin{center}
        \begin{tabular}{|*{12}{c|}}
            \hline
            $h(X)$ & 0 & 1 & 2 & 3 & 4 & 5 & 6 & 7 & 8 & 9 & 10\\ \hline
            x & 74 & & 46 & 3 & 26 & & 17 & & 58 & 57 & 32\\
            \hline
        \end{tabular}
    \end{center}

    \item Four elements would have to be probed before we find 57. This is because $57\mod11=2$ but element 46 is found at position 2. So then the double hashing algorithm is applied $7-(57\mod7)=6$. Jumping 6 buckets we find 58, jumping 6 more we find 3, jumping 6 more we find 57. $\therefore$ 57 is found on the 4\textsuperscript{th} probe.

    \item Six elements would have to be probed before we realize 68 is not in the table. This is because $68\mod11=2$ but element 46 is found at position 2. So then the double hashing algorithm is applied $7-(68\mod7)=2$. Jumping 2 buckets we find 26, jumping 2 more we find 17, jumping 2 more we find 58, jumping 2 more we find 32 and finally jumping 2 more we find null. $\therefore$ 68 is determine to not be in the hash table on the 6\textsuperscript{th} probe.

    \end{enumerate}
\end{enumerate}

\end{document}